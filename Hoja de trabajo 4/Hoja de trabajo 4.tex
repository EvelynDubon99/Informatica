\documentclass{article}
\usepackage[utf8]{inputenc}
\usepackage[left=2cm,right=2cm,top=2cm,bottom=2cm]{geometry} 
\author{Evelyn Lorena Dubón Sicán , 20181014, dubon181014@unis.edu.gt} 
\title{Hoja de trabajo No.4} 
\date{30 de agosto del 2018} 
\usepackage{natbib}
\usepackage{graphicx}
% Margins 
\topmargin= -0.45in 
\evensidemargin=0in
\oddsidemargin=0in
\textwidth=6.5in
\textheight=9.0in
\headsep=0.25in
\begin{document}
\linespread{1.1}

\maketitle
\section*{Ejercicio \#1 }
\begin{enumerate}

        \item{$a:=\{1,2,4,8,16,32,64\}$} {$\in$} {$d := \{n\in\mathtt{N}\ |\ \exists i\in\mathtt{N}\ .\ n=2^i\wedge n<100 \}$}

        \item{$b:=\{n\ \in \mathtt{N}\ |\ \exists x \in \mathtt{N}\ .\ x=n/5 \}$} {$\in$}  {$f:=\{ n\in\mathtt{N}\ |\ \exists x\in \mathtt{N}\ .\ n=x+x+x+x+x \}$}

        \item{$c:=\{n\in \mathtt{N}\ |\ \exists x\in\mathtt{N}\ .\ n=x*x \}$} {$\in$} {$e:=\{ n\in\mathtt{N}\ |\ \exists x\in \mathtt{N}\ .\ x=\sqrt{n} \}$}

\end{enumerate}
\section*{Ejercicio \#2}
\begin{enumerate}
    \item {$a:=\{n \in \mathtt{N} \ |\ \exists x \in \mathtt{N} x= n/15 $\}}
    \item {$c:= \{ n \in \mathtt{N} \ |\  \exists x \in \mathtt{N} x= n/5 \wedge x = n/4 \} $}
     \item {$d:= \{ a \forall \ 1 < x < a \ . \ a \ mod(x) \neq 0 \} $}
    \item {$b :=\{ a \ |\ a\subset P(\mathtt{N}) \ |\ \exists x \in \mathtt{N} \ .\ x/15 \ .\ \exists n \subset a \ .\ x=n/15\} $}
    \item {$d :=\{ b \ |\ b\subset P(\mathtt{N}) \ |\ \exists x \in \mathtt{N} \ .\ x + x = 42\} $}
\end{enumerate}
\section*{Ejercicio \#3}
$Q(m):= \{ n \in \mathtt{N} \ |\ n \ \leq \  m \ \wedge \ mcd \ \langle m , n \rangle = 1  \}$ \\
 $mcd= \frac{m}{n}$
\section*{Ejercicio \#4}
\begin{enumerate}
    \item $ \ x \in \ \mathtt{N} \ . \ x + x = \{ \langle x , x+x \rangle \ |\ x \in \ \mathtt{N} \} $
    \item $\ x \in \ \mathtt{N} true =\{ \langle x , true \rangle \ |\ \frac{x}{5}\} $ $\cup$ $ \ x \in \ \mathtt{N} false =\{ \langle x , false \rangle \ |\ \neg \frac{x}{5}\}$
    \item $f\circ g \ \in P(N)$
    \item $\{ \langle x, f(gx) \rangle \ |\ x \in N \wedge f(x) \in N   \wedge g(x) \subset f(x)$
\end{enumerate} 
\section*{Ejercicio \#5}
\begin{enumerate}

\item{$f(x)=x^2$} surjectiva 

        \item{$g(x)=\frac{1}{cos(x-1)}$} injectiva

        \item{$h(x)=2x$} bijectiva

        \item{$w(x)=x+1$} bijectiva
\end{enumerate}
\section*{Ejercicio \#6}
 \begin{enumerate}
 \end{enumerate}
\end{document}

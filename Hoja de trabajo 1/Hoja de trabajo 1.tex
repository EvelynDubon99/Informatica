\documentclass{article} 
\usepackage[utf8]{inputenc}
\usepackage[left=2cm,right=2cm,top=2cm,bottom=2cm]{geometry} 
\author{Evelyn Lorena Dubón Sicán , 20181014, dubon181014@unis.edu.gt} 
\title{Hoja de trabajo No.1} 
\date{25 de julio del 2018} 
\usepackage{natbib}
\usepackage{graphicx}
% Margins 
\topmargin= -0.45in 
\evensidemargin=0in
\oddsidemargin=0in
\textwidth=6.5in
\textheight=9.0in
\headsep=0.25in
\begin{document}

\maketitle

\section{Ejercicio 2: Abstracción}
\begin{itemize}
    \item El conjunto de nodos del grafo: \[ 
\left \{
  1, 2, 3, 4, 5, 6
\right \}
\]
    \item  El conjunto de vertices del grafo:\[ 
\Bigg \{
  \begin{tabular}{cccc}
\big\langle1,2\big\rangle & ,\big\langle2,6\big\rangle &,\big\langle6,5\big\rangle &, \big\langle5,1\big\rangle \\
\big\langle1,5\big\rangle & ,\big\langle5,6\big\rangle &,\big\langle6,2\big\rangle &, \big\langle2,1\big\rangle \\
\big\langle1,3\big\rangle & ,\big\langle3,6\big\rangle &,\big\langle6,4\big\rangle &, \big\langle4,1\big\rangle \\
\big\langle1,4\big\rangle & ,\big\langle4,6\big\rangle &,\big\langle6,3\big\rangle &, \big\langle3,1\big\rangle \\
  \end{tabular}
\Bigg \}
\] 
\end{itemize}

\section{Ejercicio 3}
\begin{itemize}
    \item ¿Que estructura de datos podria representar un lanzamiento de dados? \par
    La estructura de datos que lo representa es de "caminos"
    \item ¿Que algoritmo podriamos utilizar para generar dicha estructura? \par
    Se tendra un algorimo que contegna un camino para que se pueda ir teniendo un manejo de condiciones.
    \item  ¿Como nos aseguramos que ese algoritmo siempre produce un resultado? \par
    Nos podemos asegurar de que siempre tendra un diferente resultado debido a que no se utilizara ningun ciclo. 
\end{itemize}
\end{document}
